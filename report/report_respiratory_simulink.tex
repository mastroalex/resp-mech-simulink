\documentclass[12 pt,a4paper,twocolumn]{article}
\usepackage[utf8]{inputenc}
\usepackage[T1]{fontenc}
\usepackage[italian,english]{babel}
\usepackage{indentfirst} 
\usepackage{graphicx}
\usepackage{tabularx}
\usepackage{siunitx}
\usepackage{amsmath,stix,bm}
\usepackage{eucal}
\usepackage{caption}

\usepackage{multicol}
\usepackage[includeheadfoot,margin=0.7in,top=0.3 in,bottom=0.35in]{geometry}
%Per grafica vettoriale tramite InkScape%
\usepackage{color}
\usepackage{transparent}
\graphicspath{{img/}}
\usepackage[dvipsnames]{xcolor}
\usepackage{pdfpages}
\usepackage{pgfplots}
\usepackage{textcomp}

\usepackage{xcolor,colortbl}
\usepackage{listings}
\usepackage{cleveref}
\usepackage{caption}
\DeclareCaptionFont{quack}{}
\captionsetup[figure]{font={color=gray,small},labelfont={color=black,sc}}
\captionsetup[table]{font={color=gray,small},labelfont={color=black,sc}}
\captionsetup[subfigure]{font={color=gray,small},labelfont={color=black}}
\addto\captionsenglish{\renewcommand{\figurename}{Fig.}}
\addto\captionsenglish{\renewcommand{\tablename}{Tab.}}
\addto\captionsitalian{\renewcommand{\tablename}{Tab.}}
\crefname{table}{Tab.}{Tabs.}  

\usepackage{cancel}

\usepackage{subcaption}
\usepackage{titlesec}
\titleformat*{\section}{\Large\bfseries\color{myGeneralColor}}
\titleformat*{\subsection}{\large\bfseries\color{myGeneralColor}}
\titleformat*{\subsubsection}{\itshape\bfseries\color{myGeneralColor}}
\definecolor{burntsienna}{rgb}{0.91, 0.45, 0.32}
\definecolor{carrotorange}{rgb}{0.93, 0.57, 0.13}
\definecolor{darktangerine}{rgb}{1.0, 0.66, 0.07}
\definecolor{deepsaffron}{rgb}{1.0, 0.6, 0.2}
\definecolor{flax}{rgb}{0.93, 0.86, 0.51}
\definecolor{lava}{rgb}{0.81, 0.06, 0.13}
\usepackage{pifont}% http://ctan.org/pkg/pifont
\newcommand{\cmark}{\ding{51}}%
\newcommand{\xmark}{\ding{55}}%
\definecolor{mintbg}{rgb}{.63,.79,.95}
\graphicspath{{figures/}{code/figures/}{../code/figures/}} %Setting the graphicspath
\makeatletter
\def\input@path{{figures/}{code/figures/}{../code/figures/}}
\makeatother
\usepackage{import}
\usepackage{authblk}
\pgfplotsset{compat=newest}
\pgfplotsset{plot coordinates/math parser=false}
\newlength\figureheight
\newlength\figurewidth
\usepackage{placeins}
\usepackage{float}
\usepackage{blindtext}
\usepackage{authblk}
\renewcommand\Affilfont{\tiny \color{gray}}

\title{\vspace*{10 pt}\color{myGeneralColor}\Huge\textbf{Simulazione di un modello circuitale per la meccanica respiratoria in Simulink}\vspace*{1 pt}}
\author[]{Mastrofini Alessandro}
\affil[]{\small alessandro.mastrofini@alumni.uniroma2.eu}
\renewcommand*{\Authand}{ e }
\date{}
\usepackage{fancyhdr}
\pagestyle{fancy}
\fancyhf{}
	\lhead{\small\color{gray} University of Rome Tor Vergata - MSSF}
\fancyfoot[C]{\small{\thepage\  di \pageref{LastPage}}}
\renewcommand{\headrulewidth}{0pt}

\fancypagestyle{plain}{
	\renewcommand{\headrulewidth}{0pt}
	%\setlength{\headheight}{80 pt} 
	\lhead{\small\color{gray} Modellazione e Simulazione di Sistemi Fisiologici  \\
	Docente: Caselli, Federica \\
	Università degli Studi di Roma Tor Vergata\\
	Ingegneria Medica - 2022}
	\rhead{\includegraphics[height=45pt]{logo.png} }
	\fancyfoot{}
}
\usepackage{lastpage}
\addtocontents{toc}{\protect\setcounter{tocdepth}{0}}
\usepackage{lipsum}
\usepackage[
backend=bibtex,
style=numeric,
sorting=none
]{biblatex} %Imports biblatex package
\addbibresource{mybib.bib} %Import the bibliography file

\usepackage{listings}
\definecolor{codegreen}{rgb}{0,0.6,0}
\definecolor{codegray}{rgb}{0.5,0.5,0.5}
\definecolor{codestring}{rgb}{0.623, 0.176, 0.588}
\definecolor{backcolour}{rgb}{0.96,0.96,0.96}
\definecolor{bbcolour}{rgb}{0.01,0.03,0.35}
\definecolor{indexcolour}{rgb}{0,0.4,0.4}
\definecolor{myOrange}{rgb}{0.933, 0.313, 0.066}
\definecolor{myBlue}{rgb}{0, 0.298, 0.8}
\lstdefinestyle{mystyle}{
	backgroundcolor=\color{backcolour},   
	commentstyle=\color{codegreen},
	classoffset=1,
	keywordstyle=\color{bbcolour},
	numberstyle=\tiny\color{codegray},
	stringstyle=\color{codestring},
	basicstyle=\ttfamily\small,
	breakatwhitespace=false,  
	breaklines=true,                 
	captionpos=b,                    
	keepspaces=false,                 
	numbers=left,                    
	numbersep=3pt,                  
	showspaces=false,                
	showstringspaces=false,
	showtabs=false,                  
	tabsize=2
}
\lstset{texcl=false, mathescape=true,style=mystyle}
\lstset{emph={%  
		i, j,X%
	},emphstyle={\color{bbcolour}}%
}%
\definecolor{myGeneralColor}{rgb}{0, 0.227, 0.580}
\usepackage{tikz}
\usetikzlibrary{fit}
\usetikzlibrary{shapes.geometric, arrows}
\tikzstyle{startstop} = [rectangle, rounded corners, minimum width=7cm, minimum height=1cm,text centered,  fill=backcolour,text width=6.5 cm,draw=gray]
\tikzstyle{startstop2} = [rectangle, rounded corners, minimum width=5cm, minimum height=2cm,text centered,  fill=backcolour,draw=gray]
\tikzstyle{io} = [trapezium, trapezium left angle=80, trapezium right angle=100, minimum width=7cm, minimum height=1cm, text centered,text width=6.5 cm,  fill=backcolour,draw=gray]
\tikzstyle{io2} = [trapezium, trapezium left angle=70, trapezium right angle=110, minimum width=2cm, minimum height=1cm, text centered,  fill=backcolour,draw=gray]
\tikzstyle{process} = [rectangle, minimum width=7cm, minimum height=1cm, text centered,text width=6.5cm,  fill=backcolour,text badly centered,draw=gray]
\tikzstyle{decision} = [diamond, minimum width=1cm, minimum height=1cm, text centered,  fill=backcolour,draw=gray,text width=2cm]
\tikzstyle{arrow} = [thick,->,>=stealth]
\tikzstyle{process2} = [rectangle, minimum width=3.5cm, text width=3.2cm,minimum height=1cm, text centered, dashed, fill=backcolour,draw=gray]
\tikzstyle{process3} = [rectangle, minimum width=4cm,text width=4cm, minimum height=1cm, text centered,  fill=backcolour,draw=gray]
\tikzstyle{process4} = [rectangle, minimum width=6cm,text width=5.5cm, minimum height=1cm, text centered,  fill=backcolour,draw=gray]

\begin{document}


\twocolumn[{
\begin{@twocolumnfalse} 
		\vspace*{20 pt}
	\begingroup
	\let\center\flushleft
	\maketitle
	\let\endcenter\endflushleft
	\endgroup
	\begin{abstract}
% \noindent
% \noindent\\
% \noindent
% \noindent
% \noindent
La ventilazione polmonare è un'operazione critica sia durante gli interventi chirurgici che per la ventilazione assistita o sostitutiva del paziente. 

In questo report si propone l'analisi di un modello di meccanica polmonare tramite un'analogia elettrica con un circuito equivalente composto da resistenze e capacitori.
Tale analogia elettrica-meccanica permette di analizzare come la variazione dei parametri polmonari influenza la ventilazione nel caso di ventilatori a controllo di pressione e quindi di simulare la risposta del sistema polmonare alla ventilazione.

Viene inoltre proposto un modello matematico per l'implementazione di un ventilatore a controllo di pressione con onda quadra, all'interno dell'ambiente di modellazione \texttt{Simulink}, con interfaccia grafica e parametri selezionabili dall'utente. 

Vengono quindi simulate diverse condizioni patologiche andando a vedere come la variazione della compliance polmonare ha l'effetto predominante. Piccole variazioni dei valori di capacità si riflettono in grandi variazioni del volume d'aria in ingresso fino a stati pericolosi per il paziente stesso. 

%% ABSTRACT 

	\end{abstract}
	\vspace*{20 pt}
\end{@twocolumnfalse}
}]

\section{Introduzione}

% \noindent
\textcolor{blue}{
	\lipsum[1-2]
}

\section{Background}


SCRIVI QUALCOSA

\subsection{Analogia circuitale}

\begin{figure*}[t!]
	\begin{subfigure}{0.5\linewidth}
		\centering
		\small{
			\def\svgwidth{\linewidth}
			\input{circuit.pdf_tex}}
		\caption{}
		\label{fig:modello}
	\end{subfigure}\hfill
	\begin{subfigure}{0.5\linewidth}
		\centering
		\small{
			\def\svgwidth{0.7\linewidth}
			\input{lung.pdf_tex}}
		\caption{}
	\end{subfigure}\hfill
	\caption{Analogia circuitale della meccanica respiratoria \cite{khoo_physiological_2018} (a); Rappresentazione schematica della divisione del circuito polmonare in due contributi resistivi (vie aeree superiori e inferiori) e in due contributi capacitivi (compliance del polmone e della parete toracica), raffigurate anche la pressione alveolare e pleurica.}
\end{figure*}

Il circuito polmonare può essere analizzando facendo un'analogia con i circuiti elettrici.

In particolare è possibile fare un parallelismo tra il flusso d'aria e la corrente elettrica (flusso di cariche) vedendo e la pressione come la presenza di un potenziale elettrico.
Si rivede allora la resistenza meccanica come il rapporto tra l'incremento di pressione rispetto il flusso, analoga alla resisitenza elettrica. Similmente la compliance non è altro che il rapporto tra l'aumento di volume e l'aumento di pressione, in analogia elettrica è un condensatore.

Il sistema in \cref{fig:modello} è un modello di meccanica respiratoria che trascura la presenza di contributi inerziali (non ci induttanze) e considera la presenza di due compartimenti. Sono separate le vie aeree superiori, con il loro contributo resistivo $R_C$ dalle vie aeree inferiori $R_P$. I due compartimenti sono in serie tra loro ed in serie ai serbatoi d'aria, ovvero le capacità rappresentanti il contributo di compliance della parete $C_W$ e del polmone $C_L$. 
Tali contributi sono in serie proprio perchè il volume d'aria passante è lo stesso. 

A questo si aggiunge anche la capacità di shunt $C_S$ che tiene conto di diversi contributi quali lo spazio morto anatomico, la deformabilità delle vie aeree e la comprimibilità dell'aria. 

Si identificano allora anche le pressioni nei nodi. La pressione alle vie aeree $P_{aw}$, la pressione pleurica $P_{pl}$ e la pressione alveolare $P_A$. Chiaramente l'ingresso del sistema, dato dalla bocca e dalle cavità nasali, è rappresentato dalla pressione all'apertura delle vie aeree $P_{aO}$. 



\subsection{Risposta del sistema}

Il circuito in \cref{fig:modello} può essere descritto dalle seguenti equazioni:

\begin{equation}
	\footnotesize{
	\left\{\begin{array}{l}
		P_{a O}=Q R_{C}+\frac{1}{C_{S}} \int\left(Q-Q_{A}\right) \\
		\frac{1}{C_{s}} \int\left(Q-Q_{A}\right)=Q_{A} R_{P}+\left(\frac{1}{C_{L}}+\frac{1}{C_{W}}\right) \int Q_{A}
	\end{array}\right.}
\end{equation}

Si ottiene allora la funzione di trasferimento del sistema:

\begin{equation}
		\footnotesize{
\begin{aligned}
	H(s)&=\frac{Q(s)}{P_{a O}(s)}\\
	&=\frac{s^{2}+s \frac{1}{R_{P}}\left(\frac{1}{C_{S}}+\frac{1}{C_{e q}}\right)}{s^{2}\left(R_{C}\right)+s\left(\frac{R_{C}+R_{P}+\frac{R_{C} C_{S}}{C_{e q}}}{C_{S} R_{P}}\right)+\frac{1}{C_{e q} C_{S} R_{P}}}
\end{aligned}}
\end{equation}

Dove si esprime la serie delle capacità come ${1\over C_{eq}}={1\over C_{L}}+{1\over C_{W}}$. 

\subsection{Proprietà del sistema}

I coefficienti numerici vengono selezionati da \citeauthor{khoo_physiological_2018} \cite{khoo_physiological_2018}, sono riportati in \cref{tab:coefficienti}.


\begin{table}[!ht]
	\centering
	\begin{tabular}{|c|c|c|}
		\hline
		Parametro & Valore & Unità \\ \hline
		$R_C$ & 1 & H\textsubscript{2}O s / L \\ \hline
		$R_P$ & 0.5 & H\textsubscript{2}O s / L \\ \hline
		$C_L$ & 0.2 & L cm / H\textsubscript{2}O \\ \hline
		$C_W$ & 0.2 & L cm / H\textsubscript{2}O \\ \hline
		$C_S$ & 0.005 & L cm / H\textsubscript{2}O \\ \hline
	\end{tabular}
\caption{Coefficienti numerici per il sistema \cite{khoo_physiological_2018}}
\label{tab:coefficienti}
\end{table}

\section{Modellazione del sistema}

\subsection{Simulink}



\section{Conclusioni}


%\pagebreak
\section*{Disponibilità dei dati}

Il materiale è disponibile alla repository online del progetto: \url{https://github.com/mastroalex/resp-mech-simulink}

\subsection*{Codice}

\raggedbottom

\pagebreak
\printbibliography[title=Riferimenti]
%\section*{References}





\end{document}